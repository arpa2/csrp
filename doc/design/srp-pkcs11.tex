\documentclass[a4paper,11pt]{article}

\usepackage{xspace}
\def\pkcs11{PKCS$\;$\#11\xspace}
\def\srp11{SRP$\;$\#11\xspace}
\def\c_#1 {\textsf{C\_#1}\xspace}

%TODO% Contextual spacing is off
\def\tol#1{\longleftarrow\strut{}#1\strut\phantom{\longrightarrow}}
\def\tor#1{\phantom{\longleftarrow\strut}#1\strut\longrightarrow}

\parindent0pt
\parskip0.5\baselineskip

\begin{document}

\title{\srp11:\break Secure Remote Passwords behind \pkcs11}
\author{Rick van Rein\\OpenFortress}
\maketitle

\begin{abstract}
\noindent\em Secure Remote Passwords provide a zero-knowledge proof system for password-protected service access; furthermore, the protocol establishes a session key with forward secrecy properties.  These properties make it a desirable alternative to many other password schemes; its integration as a TLS cipher suite also makes it practically usable.
In this paper, we find what is required to use SRP with a randomly generated password that is protected by a \pkcs11 token.
\end{abstract}

\section{Introduction}

The SRP formalism is related to Diffie-Hellman, but it is not the same.  This means that implementation on a \pkcs11 token, with its stringent assumptions on the use of DH and the accompanying protection of secrets, is not trivial.  In fact, as will be shown here, we need to modify the client side of the protocol to be able to use SRP.  Since only the client-side changes and the cryptographic properties of the protocol are not impacted, this is a reasonable choice to consider.

\textbf{Formalism of SRP.}
The following exchange is defined for SRP.  Client and service are assumed to agree on DH parameters,  a generator $g$ and a hash function $H$.  All equalities in this paper are modulo the DH modulus.  The password is fixated by the client as $P$, which may be shared among many services without loss of security.  For each service, the client generates a random salt $s$ and passes it with a verifier $v$ to the service while setting up an account $C$.  So, the service stores $C\mapsto\langle s,v\rangle$ and treats $v$ as a secret, because it might elicit password cracking attempts.  The client stores $C$, which it sends to the service to get back the salt $s$ that it setup in the past.
%
\begin{eqnarray*}
\textbf{Client}&\textbf{Protocol}&\textbf{Service}\\
x=H(s,P)&&\\
v=g^x&&\\
&\tor{C,s,v}&\textrm{store $C\mapsto\langle s,v\rangle$}
\end{eqnarray*}

During authentication, the client picks random value $a$ and the service picks random values $u$ and $b$.  We assume the entropy of each of these three values to match the size of the to-be-derived session key.  The authentication protocol is conducted as follows:
%
\begin{eqnarray*}
\textbf{Client}&\textbf{Protocol}&\textbf{Service}\\
&\tor{C}&\textrm{lookup $\langle s,v\rangle$}\\
&\tol{s}&\\
x=H(s,P)&&\\
A=g^a&&\\
& \tor{A}&\\
k=H(N,g)&&k=H(N,g)\\
&&B=k\cdot v+g^b\\
&\tol{B}&\\
u=H(A,B)&&u=H(A,B)\\
S=(B-k\cdot g^x)^{a+ux} && S=(A\cdot v^u)b\\
K=H(S)&&K=H(S)\\
M_1=H(A,B,K)&&\\
&\tor{M_1}&\textrm{verify $M_1$}\\
&&M_2=H(A,M_1,K)\\
\textrm{verify $M_2$}&\tol{M_2}&
\end{eqnarray*}



\textbf{Rules of the game.}  We define the following constraints to the work laid out in this paper:
\begin{itemize}
\item The purpose of using \pkcs11 is to protect credentials from extraction, so that the SRP functions of the client cannot be implemented without the \pkcs11 token holding the password.
\item The random value $a$ is not protected by \pkcs11 because it only impacts the session key, and this value is also known to the service.
\item Intermediate values will be analysed; they are considered `public', that is need no \pkcs11 protection, when they can be derived on the service-side of the protocol.  Note however, that a passive observer sees less.
\item We only define client-side computations with \pkcs11, because the the service side does not handle any secrets, thanks to the zero-proof nature of the protocol.
\item We follow the assumption of SRP that the verifier is kept a secret by the service side; this is chiefly in the interest of the service, the client could employ SRP salt pinning to enforce it.  Salt pinning falls outside the scope of this paper.
\end{itemize}

\section{First Naive Implementation}

To demonstrate the problems that arise when implementing unchanged SRP on a \pkcs11 interface, we start off by making a sound technical implementation, and point out where this fails to satisfy the desired password protection for which \pkcs11 is being considered.

\textbf{Password establishment.}
A new password is generated with the DH mechanism in \c_GenerateKeyPair, in such a way that it must remain on the token.   This will be the password $P$ for the formalism.  Since the client performing this computation is also the party to generate salt and verifier, and to authenticate on later returning, there should be no conflict of interest to support key export.  Only for reasons of backup might wrapped export be permissible.

\textbf{Credentials generation.}
To derive service access credentials from the password, first generate a random salt $s$.  The value is public, and may be generated on or off the \pkcs11 token.  Then, the token's \c_DigestXXX functions must be used to compute the hash value $x=H(s,P)$ on the token; the password can be incorporated with \c_DigestKey pointing at $P$.

Now run \c_DeriveKey with $g$ as the base value and point to $x$ as the private key or exponent.  The outcome is $v=g^x$ which is supplied in a pair $\langle s,v\rangle$ to the service, together with a username.

\textbf{Authentication.}
The client starts off as basic DH with \c_GenerateKeyPair.  The private key will be known as $a$ and the public key is $A$.  Send $A$ over to the service.

The service generates $B=v+g^b$ for a local random secret $b$ and sends $B$ and $u$ to the client, where $u$ is another random scrambling parameter that is publicly shown.

The client now computes $B-g^x = B-v$, which is entirely based on public values.  The result equals $g^b$, which can be combined with session key $a$ in \c_DeriveKey to find $g^{ab}$, a normal DH result.

Then, $B^u$ is computed from public values, and used as a base key in \c_DeriveKey, with DH private key $x$.  This yields $B^{ux}=g^{bux}$, a value that can also be calculated separately on the service, as $v^{bu}=g^{xbu}$.

The values $g^{ab}$ and $g^{bux}$ are now multiplied outside of \pkcs11 to find shared secret $S$, which is of course also derived on the service.  Likewise, the session key $K=H(S)$ is also available to the service and not considered a secret that would benefit from \pkcs11 protection.

\textbf{Problem Analysis.}
The value of this implementation is that it never needs to extract the password $P$, so \pkcs11 can be exploited as a barrier to the secret.  However, this implementation calculates $x$ from a \c_DigestFinal invocation, which means it is delivered into a byte array that is extracted from \pkcs11.  The value $x$ is the only part of the protocol that contains $P$, so knowing $x$ is as useful as knowing $P$, at least for a service that uses the given salt value $s$.

In fact, in terms of the original AKA formalism of which SRP is an instance, the computation of $x$ falls outside the formalism as a mere parameter.  The use of $x$ in the SRP protocol may be considered a technical aspect, rather than a cryptographic necessity.  This means that we may consider modifications to the calculation of $x$ to make it better suited for \pkcs11 implementation.

\section{Client-side Variation on the Formalism}

The proposed variation to better adapt SRP to \pkcs11 is to use another one-way function than $H(s,P)$.  Within the SRP formalism, the operations for the DH mechanism offers an alternative, namely the \pkcs11 function \c_DigestKey which can be used as a modular-exponentiation operation, possibly incorporating secret exponentials.

As an alternative to $x$ in plain SRP, we define $x'=H(s) P$ to improve the integration.  This looks awful, because it is possible to derive $P$ from the values of $x'$ and $s$, but the damage is controlled by not actually depending on the value $x'$ in the implementation; instead, we use it to modify the formalism.  Note that the hash function has no cryptographic role in the AKA formalism, so we might even have specified $sP$ instead of $H(s) P$.  We choose to retain the hash because the salt may be used to carry additional values to pin down the service, and that could be computationally expensive or run into size constraints without the hash function.

During secret establishment, the client does not generate a random $P$ on the token, but instead constructs a DH key pair $\langle p,P\rangle$:
%
\begin{eqnarray*}
p &=& g^P
\end{eqnarray*}

Both $P$ and $p$ are concealed by \pkcs11, with no permission of extracting them.  Wrapped export might be supported to facilitate backup and recovery, but the values of $p$ and $P$ are assumed to not appear in plaintext outside of \pkcs11.

When setting up a service with a verifier and salt, the following process is used:
%
\begin{eqnarray*}
\textbf{Client}&\textbf{Protocol}&\textbf{Service}\\
% x' can be chosen freely within security constraints, and then v' is g^{x'}
x'=H(s) P&&\\
v'=p^{H(s)}&&\\
&\tor{C,s,v'}&\textrm{store $C\mapsto\langle s,v'\rangle$}
\end{eqnarray*}

The service cannot tell the difference, because the values $s$ is treated as an opaque string and because it is not of a different format; the value $v'$ is also not distinguishable for the service.

During authentication, the client picks random numbers $a$ and the service picks random values $u$ and $b$.  The authentication protocol is conducted as follows:
%
\begin{eqnarray*}
\textbf{Client}&\textbf{Protocol}&\textbf{Service}\\
&\tor{C}&\textrm{lookup $\langle s,v'\rangle$}\\
&\tol{s}&\\
v'=p^{H(s)}&&\\
A=g^a&&\\
&\tor{A}&\\
k=H(N,g)&&k=H(N,g)\\
&&B=k\cdot v'+g^b\\
&\tol{B}&\\
u=H(A,B)&&u=H(A,B)\\
S'=(B-k\cdot v')^a\cdot (((B-v')^u)^{H(s)})^P && S=(A\cdot {v'}^u)^b\\
K'=H(S')&&K=H(S)\\
M_1'=H(A,B,K')&&\\
&\tor{M_1'}&\textrm{verify $M_1'$}\\
&&M_2'=H(A,M_1',K)\\
\textrm{verify $M_2'$}&\tol{M_2'}&
\end{eqnarray*}

The calculation procedure for $S'$ by the client differs from the original calculation, but ought to yield the same output as the standard calculation of $S$ by the service under proper authentication conditions.  The same then holds for $K'$, $M_1'$ and $M_2'$.  We only use the accents to the symbols to distinguish their calculation and be able to prove equality.

Note that the value $x'$ is never seen in this calculation, nor can it be derived.  The value $P$ is used, but cannot be extracted from \pkcs11.

In terms of the original work on SRP, this is still an AKA formalism.  The change from $x=H(s,P)$ to $x'=H(s) P$ occurs in the periphery surrounding the AKA formalism definitions.  We will consider efficiency and security impacts in subsequent sections, but first turn to an implementation in terms of \pkcs11, using standard DH mechanisms.

\textbf{Proof of correctness.}
During authentication, the two sides need to find the same values for $S$ and $S'$, after which the values for $K'$, $M_1'$ and $M_2'$ follow without a change.  The thing to prove is that $S'$ in the client-side formalism yields the same result as $S$ in the service-side formalism, given that the service has received the value $v'$ instead of $v$.  The correctness is proven as follows:
\begin{eqnarray*}
S' &=& (B-k\cdot v')^a\cdot (((B-k\cdot v')^u)^{H(s)})^P\\
&=& (g^b)^a\cdot (g^b)^{uH(s)P}\\
&=& g^{ba}\cdot g^{buH(s)P}\\
&=& g^{ab}\cdot ((g^P)^{H(s)})^{ub}\\
&=& (g^a)^b\cdot (p^{H(s)})^{ub}\\
&=& A^b\cdot {v'}^{ub}\\
&=& (A\cdot v'^u)^b\\
&=& S
\end{eqnarray*}
This establishes that the unmodified SRP service algorithm, as well as the unmodified networking protocol, can work with the modified client that this formalism introduces.  The one thing to care for is that all clients adhere to the same new formalism, since they will need to base their work on $v'$ and not $v$.  Since the password pair $\langle p,P\rangle$ is concealed by \pkcs11, this constraint is already implied by the technology.

\textbf{Problem Analysis.}
The formalism presented here suffers from a cryptographic problem, namely that the value $p$ is assumed to be secret.  In normal Diffie-Hellman operations, only $P$ would be considered secret, and $p$ would be a public key.  Indeed, the value of $p$ could be recomputed from $g$ and $P$ by anyone with \pkcs11 access, and extracted.  The impact of this extraction would be that new values of $v'$ could be calculated without the help of \pkcs11; only authentication would continue to rely on live access to \pkcs11, but that is not the best possible result.  We therefore construct one further iteration.

Another problem that remains is the potential relationship between values of $v$.  Consider two values of $H(s)$, called $h_1$ and $h_2$.  Since these are random values, it may happen that $h_2$ is divisable by $h_1$, so when $v_1=p^{h_1}$ then $v_2=p^{h_2}=v_1^{h_2/h_1}$.  This problem is mitigated in plain SRP through the use of the hash function, and the next iteration must also resolve it through better scattering of the verifiers.

\section{Maximally \pkcs11 Protected Formalism}

On top of the suitability for \pkcs11 in previous variation, the additional requirement is now to avoid that verifiers can be constructed from extractable values like $p$.  This is ensured by applying the secret key $P$ to the salt.

As an alternative to $x$ and $x'$, we therefore define $x''={H(s)}^P P$.  This introduces a multiplication with secret $P$, which will make it impossible to derive password-equivalent value $x''$ in the implementation.  It also raises the salt to the power $P$ to require the password before the verifier can be computed.  Under the assumption that services conceal the verifier, a new service cannot be setup with the same salt, and the dependency on $P$ implies that only the client with \pkcs11 access can thus create a new salt and verifier for a new service.

The secret key $P$ established as before, with token-generated random key $P$, protected from extraction.  But in this variation, the value $p$ does not have to be protected from extraction.
%
\begin{eqnarray*}
p &=& g^P
\end{eqnarray*}

When setting up a service with a verifier and salt, the following process is used:
%
\begin{eqnarray*}
\textbf{Client}&\textbf{Protocol}&\textbf{Service}\\
% x'' can be chosen freely within security constraints, and then v' is g^{x'}
x''={H(s)}^P P&&\\
v''=p^{{H(s)}^P}&&\\
&\tor{C,s,v''}&\textrm{store $C\mapsto\langle s,v''\rangle$}
\end{eqnarray*}

As before, the service cannot tell the difference, because the value $s$ is treated as an opaque string and because it is not of a different format; the value $v''$ is also not distinguishable for the service.

During authentication, the client picks random value $a$ and the service picks random values $u$ and $b$.  The authentication protocol is conducted as follows:
%
\begin{eqnarray*}
\textbf{Client}&\textbf{Protocol}&\textbf{Service}\\
&\tor{C}&\textrm{lookup $\langle s,v''\rangle$}\\
&\tol{s}&\\
v''=p^{{H(s)}^P}&&\\
A=g^a&&\\
&\tor{A}&\\
k=H(N,g)&&k=H(N,g)\\
&&B=k\cdot v''+g^b\\
&\tol{B}&\\
u=H(A,B)&&u=H(A,B)\\
S''=(B-k\cdot v'')^a\cdot (((B-k\cdot v'')^u)^{{H(s)}^P})^P && S=(A\cdot {v''}^u)^b\\
K''=H(S'')&&K=H(S)\\
M_1''=H(A,B,K'')&&\\
&\tor{M_1''}&\textrm{verify $M_1''$}\\
&&M_2''=H(A,M_1'',K)\\
\textrm{verify $M_2'';$}&\tol{M_2''}&
\end{eqnarray*}

The calculation procedure for $S''$ by the client differs from the original calculation, but ought to yield the same output as the standard calculation of $S$ by the service under proper authentication conditions.  The same then holds for $K''$, $M_1''$ and $M_2''$.  We only use the accents to the symbols to distinguish their calculation and be able to prove equality.

Note that the value $x''$ is never seen in this calculation, nor can it be derived.  The value $P$ is used, but cannot be extracted from \pkcs11.

In terms of the original work on SRP, this is still an AKA formalism.  The change from $x=H(s,P)$ to $x''={H(s)}^P P$ occurs in the periphery surrounding the AKA formalism definitions.  We will consider efficiency and security impacts in subsequent sections, but first turn to an implementation in terms of \pkcs11, using standard DH mechanisms.

\textbf{Proof of correctness.}
During authentication, the two sides need to find the same values for $S$ and $S''$, after which the values for $K''$, $M_1''$ and $M_2''$ follow without a change.  The thing to prove is that $S''$ in the client-side formalism yields the same result as $S$ in the service-side formalism, given that the service has received the value $v''$ instead of $v$.  The correctness is proven as follows:
%
\begin{eqnarray*}
S'' &=& (B-k\cdot v'')^a\cdot (((B-k\cdot v'')^u)^{{H(s)}^P})^P\\
&=& (g^b)^a\cdot (g^b)^{u{{H(s)}^P P}}\\
&=& g^{ba}\cdot g^{bu{H(s)}^P P}\\
&=& g^{ab}\cdot ((g^P)^{{H(s)}^P})^{ub}\\
&=& (g^a)^b\cdot (p^{{H(s)}^P})^{ub}\\
&=& A^b\cdot {v''}^{ub}\\
&=& (A\cdot v''^u)^b\\
&=& S
\end{eqnarray*}
This establishes that the unmodified SRP service algorithm, as well as the unmodified networking protocol, can work with the modified client that this formalism introduces.  The one thing to care for is that all clients adhere to the same new formalism, since they will need to base their work on $v''$ and not $v$ or $v'$.  Since the password $P$ is concealed by \pkcs11, this constraint is already implied by the technology.


\section{Implementation as \srp11}

The following describes how to implement the last iteration of the SRP mechanism and demonstrates its integration with \pkcs11.  Where the distinction is useful, we will distinghuish `plain SRP' from this modified form, which we will refer to as \srp11.

\textbf{Password establishment.}
A new `password' is generated with the DH mechanism in \c_GenerateKeyPair, leading to a randomly generated private key $P$ and a public key $p$.  Only for reasons of backup might wrapped export of $p$ be permissible, but $P$ can be handled with more relaxation.  The two are never used separately however, so there is no real use in treating them any differently.

It is not advised to delete the public key.  Although it can be derived in theory from the private key by supplying a `remote offer' valued 1 to \c_DeriveKey, the outcome of this function is not likely to be permitted having the \texttt{CKK\_DH} key type.  TODO: But we don't need that setting, as we will export the public key and use it off-token anyway, since generic mod-exp is not available on the token anyway.

The base value used is the generator $g$, to establish the relationship $p=g^P$.  Both the base value and the modulus are commonly used group parameters for the SRP scheme, with a few standardised in Appendix A of RFC 5054.

Since it is not necessary to export the value of $P$, it can be created in such a way that \pkcs11 enforces this key concealment.  The value $p$ is retained for efficiency purposes, even if it could also be derived on the fly through $g^P$ with \c_DeriveKey.

TODO:EDITED-TILL-HERE

\textbf{Credentials generation.}
To derive service access credentials from the password, first generate a random salt $s$.  The value is public, and may be generated on or off the \pkcs11 token.  Then, the token's \c_DigestXXX functions must be used to compute the hash value $x=H(s)$ either on or off the token; the outcome can be used to calculate $p^{H(s)}$ off-token.  The result of that operation is input to \c_DeriveKey, where it is raised to the power $P$ in a modular exponentiation operation on the token with \c_DigestKey.  The outcome is $v''$, which is supplied in a pair $\langle s,v\rangle$ to the service, together with a username.

\textbf{Authentication.}
The client starts off as basic DH with \c_GenerateKeyPair.  The private key will be known as $a$ and the public key is $A$.  Send $A$ over to the service.

The service generates $B=v''+g^b$ for a local random secret $b$ and sends $B$ and $u$ to the client, where $u$ is another random scrambling parameter that is publicly shown.

Through the stored values of $p$ and $P$, the client can now reconstruct $v''$, which requires the non-extractable secret $P$; moreover, the outcome cannot be reversed to $P$ due to the properties of the discrete logarithm problem.  The value $v''$ is also known to the service, so it may be exported from the \pkcs11 interface.  The client now uses $v''$ find $B-v''$, which is entirely based on values that the service also sees.  The outcome equals $g^b$, which can be combined with session key $a$ in \c_DeriveKey to find $g^{ab}$, a normal DH result.

The remainder of the computation of $S''$ consists of using the value $B-v''$ in off-token multiplication and modular exponentiation, all based on the public values $B-v''$, $u$, $H(s)$ and the locally generated random value $a$, all of which are safe to do outside of \pkcs11, and there will be two modular exponentiations with secret key $P$ as the exponent, to be conducted on the token behind \pkcs11.  It is this last point where the token is a strict requirement in deriving the proper values.

Note that the value $S''$ is also available on the service, so we consider it public.  [\textbf{TODO:} How about the $\ldots^P$ and ${\ldots^P}^P$ values, are they `public' too?]  Likewise, the session key $K=H(S)$ is also available to the service and not considered secret.

Note that the value of $x''$ is absent from the calculations.  Instead, the secret key $P$ is used.  In a normal calculation the dependency on a secret key might be a liability, but when using \pkcs11 this is not as charged because of its assumed protection of private keys.  The value of $x''$ cannot be derived either, since the protected storage of the password values make it impossible to find the outcome of $H(s)p$ because that is not part of the DH mechanism.


\section{Performance Considerations}

The use of the more protective private-key encapsulation of \pkcs11 comes at a price of reduced efficiency.  Note that this plays particularly at the client side, which means that it is subject to user selection of technology, and more importantly, meaning that scaling problems are not likely to be a problem --- as we are talking of an alternative to manual password entry.

Still, a performance indication is useful.  The overruling consideration in this sense is how many modular-exponential operations take place.  More accurately put, the number of bits that may hold a value in the exponent of these operations.  We are going to base our performance considerations on the assumption that the random values $s$, $a$, $b$, $u$ all match the size of the outcome of the hash algorithm $H$.  So, we can count the number of times such bits occur in the exponent of a calculation.

The chief concern is the authentication process.  In the original process we find client-side exponents $a$, $x$ and $a+ux$, sized like 1, 1 and 2 hashes, so plain SRP weighs like 4 modular exponentiations with the size of the outcome of $H$.  In \srp11, the client-side exponents during authentication are $H(s)$, $a$, $a$, $u$, $P$, $P$ and $H(s)$, which weighs like 7 modular exponentiations.

Clearly, the inability of computing $a+ux$ before the exponention forces in extra computations in return for the added security of not handling $x$ in \srp11.  The computational comlexity of \srp11 is about 75\% higher than for plain SRP.  This factor is probably irrelevant in comparison to the (possible) move of the calculation to a piece of hardware.  Hardware tends to be either much slower (USB token) or much faster (HSM, crypto co-processor) at performing modular exponentiation than a CPU.


\section{Security Considerations}

The values that are available as intermediate results in \srp11 are all `public', in the sense that the service can also find them based on what it knows.  Most computations are based on a generator raised to a (secret or public) exponent, which is considered irreversable on grounds of the infeasability of the discrete logarithm problem.  [\textbf{TODO:} The values $\ldots^P$ and ${\ldots^P}^P$ might be non-`public'.]

The value $v''$ is known to the service, and therefore its derivation as $p^{H(s)P}$ during authentication does not reveal information that should be protected by \pkcs11.  The value $B-v''$ is `public', and so is $a$, so computing $(B-v'')^a$ presents no problems.  The same holds for $(B-v'')^u$ and the next step, $((B-v'')^u)^{H(s)}$.  The latter value is then twice raised to the value $P$ that is protected behind \pkcs11, but an attacker would need to crack discrete logarithms to reverse this to find $P$.  The outcome of this last operation is equal to the service-side value $S$ multiplied by the inverse of the public-value based $(B-v'')^a$, so having the value once more reveals no new information.  The remainder computers $K$, $M_1$ and $M_2$ on both ends, so there is no question about their revelation of new information that ought to be protected \pkcs11.

It is worth noting that a client might turn rogue when it is cracked; in such cases, it might gain the rights to authenticate, but it should not gain access to the passwords through which it can mount future attacks on other locations.  This makes the common assumption that the credential protection of \pkcs11 is not cracked; an assumption that can be influenced by a suitable choice of \pkcs11 implementation.  Plain SRP and \srp11 are not different in this respect, and in fact it is common to all DH mechanism.  Another point common to all three variations is that thrashing of the random values $a$ and $b$ suffices to thwart future attackers from recovering session keys of traffic preceding the crack on the client (forward secrecy).

The value $p=g^P$ is retained as a `cache' value.  Rogue access to \pkcs11 might consider replacing this value with a more suitable value, to influence future computations based on it.  This is not likely to lead to anything useful however; first, it would be noticed when accessing a non-rogue service; second, the value $P$ is still used in computations; and third, the discrete logarithm problem makes it notoriously difficult to derive a value of $p$ that would lead to a desired outcome.  A much better attack would be to replace both $p$ and $P$ which is comparable to replacing a password; this however, would not leak any information about the prior value of $P$ and would invalidate access to services setup with it.

Modular exponentiation cannot be reversed, as a result of the discrete logarithm problem.  The start of the computation is $B-v''$, which equals $g^b$ and that uses an unknown value $b$.  Raising this to the powers $u$, $H(s)$, $P$ and $P$ in any order should stir the value even further, but in lieu of the value $b$ on the client and due to the discrete logarithm problem it is unlikely that an attack could be mounted from this angle.  \textbf{Open Question:} Can we prove that?

In conclusion, the maximum protection that \pkcs11 can deliver has been achieved with \srp11, under the assumption that what the service knows is sufficiently public.  That angle however, is subject of potential crisitism.  It could be argued that the service is in a special position.  For one, it generates random string $b$ internally and only publishes the corresponding $B$; but that alone would not make much difference, an active MITM attack could produce similar values. The vital distinction is that the service knows $v'$, which is assumed to have been relayed to it in a covert manner, and so it is a shared secret between client and service --- one that would not be available to anyone on the channel between client and service.

Unfortunately, it is not possible to redesign \srp11 to treat $v''$ as a secret value.  This can be seen directly in the computations; the construction $(B-v'')^e$, regardless of the exponent $e$ cannot be computed with DH values, because subtraction is not foreseen in this mechanism.  This is where \pkcs11 reveals that it was not designed with plain SRP or \srp11 in mind.  And since exponentiation does not distribute through subtraction, neither modular or otherwise, we cannot construct any such mechanism cleverly.

Note that the visibility of $v''$ is less than perfect, but not truly problematic in itself.  The sole reason for treating all data available to the service as public is to avoid putting too much faith in what \pkcs11 can do.  Apparantly, this is its limitation.  Note that this is also caused by something else --- the client is the party that originally generated $v''$ and would be able to do that again.

In conclusion, \srp11 actually improves the security of SRP by taking the passwords out of sight and placing them behind \pkcs11; in addition, the password-equivalent value $x$ is not available to the \srp11 computations due to \pkcs11 shielding.  The one thing that is not possible, is to also conceal the computation of verifier values $v''$ from the client, but it has been demonstrated that this would have introduced a false sense of security anyway.  In conclusion, \srp11 establishes precisely the maximum achievable and the maximum desirable protection of SRP credentials.


\section{Conclusions}

This article introduces \srp11, a variation on SRP that uses \pkcs11 to protects the password and password-equivalent values that are computed on the client side.  It has been shown that a direct implementation of plain SRP with \pkcs11 does protect the original password, but not the password-equivalent value $x$, which means it is useless.  In contrast, the \srp11 variation achieves precisely the maximum achievable and maximum desirable that \pkcs11 could offer.

It has been shown that the modified generation of salt and verifier combined with the modified client-side authentication computations compute values equivalent to those of plain SRP, and that the service or protocol require no changes.  In other words, only the client needs to modify its computations if it intends to use \pkcs11.

In terms of efficiency, the \srp11 mechanism adds 75\% to the computational complexity, relative to plain SRP.  When \pkcs11 is implemented on hardware, then it is likely that this choice has an overruling impact on the speed of the computations --- they would take longer when a plain USB token is used, and would go down when cryptographic acceleration is built into the hardware.

\textbf{Open Issue:} It remains to be seen if the details of the DH mechanism in \pkcs11 will indeed permit the repeated use of modular exponentiation.  An attempt to implement appears to be a useful exercise, in light of the potential achievements of \srp11; if this mechanism does not work, then chances of finding another that will work are modest, to put it mildly.

\end{document}
